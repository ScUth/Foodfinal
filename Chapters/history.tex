\chapter{History}
\fontsize{16pt}{16pt}\normalfont

Straw mushrooms have been cultivated for thousands of years in Asia, but the earliest record of cultivation dates back to the 18th century. Buddhist monks at the Nanhua Temple in China grew the fungi on paddy straw for the mushroom's high nutritional properties and also used it in traditional medicine. Through exposure at the temple, Straw mushrooms widely increased in popularity across China and even became a gift that was given to Chinese royalty. It has been spread across Southeast Asia, remaining primarily in the areas that they are grown in due to their short shelf life and delicate nature when fresh.
\medbreak
In the modern-day, Straw mushrooms have remained one of the most popular varieties consumed throughout Asia and are cultivated on many different agricultural waste substrates. Besides straw, the mushrooms are grown on cotton waste known locally as `gin trash'. This substrate is the fiber matter left after cotton is extracted for commercial use. Straw mushrooms are also grown on compost piles, grass, leaves, and wood chips, and can be found growing naturally on termite mounds in Southeast Asia. Straw mushrooms still grow wild in Asia and are also cultivated on a small scale in the Philippines, Malaysia, Thailand, Vietnam, China, and Eastern Europe. Outside of Asia, the mushrooms are available in canned and dried form in Western Europe, North America, and Australia.
\medbreak
\begin{figure}[h]
    \includegraphics[width=7cm]{Picture1 (2).jpg}
\end{figure}
Straw mushrooms, botanically classified as Volvariella volvacea, are small, edible fungi with a mild, musky flavor that belong to the Pluteaceae family. Also known as Chinese mushrooms, Paddy Straw mushrooms, and Nanhua mushrooms. Straw mushrooms are widely consumed in Asia and are valued for their neutral flavor, versatility, and high nutritional properties. Straw mushrooms are cultivated in the warm, tropical climates of Asia and are often grown on agricultural wastes such as rice straw, which is where the mushroom also earned its name. The fungi can be harvested in its young or mature state, with the young, unopened mushrooms being labeled as unpeeled and the opened mushrooms labeled as peeled. Unpeeled mushrooms are the most popular version sold in local markets in Asia as they are believed to have higher nutritional properties and a stronger flavor. It is important to note that Straw mushrooms are primarily found in Asia, and in North America, there is a highly toxic look-alike known as the death cap or amanita phalloides that can be lethal when consumed.\par
\begin{figure}[h]
    \includegraphics[width=7cm]{Picture2 (2).png}
\end{figure}