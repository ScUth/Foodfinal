\chapter{How to grow}
\section{What You Need to Grow Straw Mushrooms}
\subsection{Straw}
\begin{itemize}
    \item \textbf{Type}: Choose clean, dry, and pesticide-free straw.
    \begin{enumerate}
        \item \textbf{Recommended options}\\ Rice or wheat straw are ideal due to their readily digestible cellulose content. Other options like oat straw or barley straw can also work, but may require slightly longer soaking times.
        \item \textbf{Quantity} \\ Aim for approximately 1 kg (2.2 lbs) of straw for a single cultivation cycle. This amount can be adjusted based on the size of your container and desired yield.
    \end{enumerate}
\end{itemize}
\subsection{Mushroom Spawn}
\begin{itemize}
    \item \textbf{Species}: urchase high-quality \textit{Volvariella volvacea} spawn from a reputable supplier. This ensures you're getting the specific fungal strain suitable for growing straw mushrooms.
    \item \textbf{Form}: pawn typically comes in various forms like grain spawn or sawdust spawn. Both options work well, but grain spawn may colonize the straw slightly faster due to its readily available nutrients.
    \item \textbf{Quantity}: Generally, a spawn-to-substrate ratio of 1:10 is recommended. For 1 kg (2.2 lbs) of straw, you'll need around 100 g (3.5 oz) of spawn.
\end{itemize}
\subsection{Container}
\begin{itemize}
    \item \textbf{Type}: Choose a large container with good ventilation to allow for air circulation and prevent moisture buildup. Plastic tubs, buckets, or even large grow bags with ventilation holes can work well.
    \item \textbf{Size}: The size of the container will depend on the amount of straw you're using. Aim for a container that can comfortably hold the straw without being crammed, allowing for some space for air circulation. A 50 L (13 gallons) container is a good starting point for 1 kg (2.2 lbs) of straw.
\end{itemize}
\subsection{Hydrated Lime}
\begin{itemize}
    \item \textbf{Function}: Adding hydrated lime (calcium hydroxide) to the soaking water helps regulate the pH level of the straw, creating a slightly alkaline environment that favors the growth of \textit{Volvariella volvacea} while suppressing potential contaminants.
    \item \textbf{Quantity}: Use approximately 50 g (1.75 oz) of hydrated lime per liter (gallon) of water during the soaking process.
\end{itemize}
\subsection{Spray Bottle}
\begin{itemize}
    \item \textbf{Purpose}: A clean spray bottle filled with water will be essential for maintaining humidity inside the container throughout the growing process.
\end{itemize}
\subsection{Thermometer}
\begin{itemize}
    \item Monitoring the temperature is crucial for optimal mushroom growth. A thermometer will help you ensure the environment stays within the ideal range for \textit{Volvariella volvacea} (25-30$^\circ$C / 77-86$^\circ$F).
\end{itemize}

\section{Step to grow}
\subsection{Prepare the Straw}
\begin{itemize}
    \item \textbf{Chopping}: Cut the straw into small pieces, ideally between 3-5 cm (1-2 inches) in length. This size provides optimal surface area for efficient fungal colonization while maintaining good air circulation within the substrate.
    \item \textbf{Soaking}: Submerge the chopped straw in a large container filled with lukewarm water (around 30$^\circ$C / 86$^\circ$F) for 24-48 hours. This process hydrates the straw, making it easier for the fungal mycelium to colonize and absorb nutrients.
    \item \textbf{Adding Hydrated Lime}: During soaking, add approximately 50 g (1.75 oz) of hydrated lime per liter (gallon) of water. The slightly alkaline environment created by lime helps suppress the growth of competing bacteria and fungi while favoring the growth of \textit{Volvariella volvacea}.
    \item \textbf{Draining}: After the soaking period, thoroughly drain the straw using a colander or mesh sieve. Squeeze out excess water gently, aiming for the straw to be moist but not dripping. Excessive moisture can lead to contamination and hinder fungal growth.
\end{itemize}
\subsection{Pasteurization (Optional)}
\begin{itemize}
    \item \textbf{Purpose}: Pasteurization is an optional step that helps eliminate potential contaminants like bacteria and mold spores that may be present in the straw. While not strictly necessary for all situations, it can improve the chances of successful cultivation, especially for beginners.
    \item \textbf{Methods}: There are two common methods for pasteurizing straw:
    \begin{enumerate}
        \item \textbf{Submersion}\\ Bring a large pot of water to a boil. Place the straw in a heat-resistant mesh bag or colander and submerge it in the boiling water for 1-2 hours. Maintain a rolling boil throughout the process.
        \item \textbf{Steaming}\\ Spread the straw on a baking sheet and steam it for 1-2 hours using a steamer or pot with a steamer basket. Ensure adequate steam is generated and reaches all parts of the straw.
    \end{enumerate}
\end{itemize}
\subsection{Inoculate the Straw}
\begin{itemize}
    \item \textbf{Spreading the Straw}: Evenly distribute the cooled and prepped straw in the chosen container. Aim for a loose and fluffy arrangement to allow for proper air circulation and prevent compaction.
    \item \textbf{Adding Spawn}: Break up the mushroom spawn into small pieces and sprinkle it evenly over the surface of the straw. Ensure good distribution throughout the substrate.
    \item \textbf{Mixing}: Gently mix the top layer of straw with the spawn, incorporating it slightly without disturbing the overall structure of the substrate. This ensures close contact between the spawn and the straw, facilitating fungal colonization.
\end{itemize}
\subsection{Create a Humid Environment}
\begin{itemize}
    \item \textbf{Covering}: Cover the container loosely with a plastic bag or cloth that allows for some air exchange. This helps trap moisture inside and maintain high humidity levels necessary for fungal growth.
    \item \textbf{Misting}: Regularly mist the inside of the container with clean water using the spray bottle. Aim for a fine mist that creates a humid environment without saturating the straw.
\end{itemize}
\subsection{Maintain Ideal Conditions}
\begin{itemize}
    \item \textbf{Temperature}: Place the container in a warm location with temperatures between 25-30$^\circ$C (77-86$^\circ$F). This temperature range is optimal for the growth of \textit{Volvariella volvacea}. If needed, use a heat mat placed underneath the container to maintain consistent warmth.
    \item \textbf{Humidity}: Monitor the humidity level inside the container and maintain it around 80-90\% by misting regularly and adjusting the ventilation as needed. Too much humidity can lead to mold growth, while insufficient moisture can hinder fungal development.
    \item \textbf{Lighting}: Avoid exposing the container to direct sunlight. Straw mushrooms do not require light for growth and may even be inhibited by excessive light exposure.
\end{itemize}
\subsection{Incubation and Fruiting}\
\begin{itemize}
    \item \textbf{Mycelial Colonization}: Allow the container to remain undisturbed for 7-10 days. During this incubation period, the fungal mycelium will colonize the straw, spreading throughout the substrate as white threads become visible.
    \item \textbf{Fruiting Body Formation}: Once the straw is fully colonized, small pinheads will begin to form on the surface, indicating the initiation of fruiting body development. This typically takes another 3-5 days.
\end{itemize}
\subsection{Harvest and Enjoy}
\begin{itemize}
    \item \textbf{Maturity}: Harvest the mushrooms when the caps are fully expanded but before the veil breaks (the thin membrane connecting the cap to the stem). This ensures optimal flavor and texture.
    \item \textbf{Harvesting Technique}: Gently twist the mushrooms at the base to detach them from the substrate. Avoid pulling or cutting, as this can damage.
\end{itemize}